\section{Related Work}
There are many definitions of what a narrator is. While many of these definitions are similar, they each have their own different nuances. Abbot describes a narrator as "One who tells a story"\cite[p. 238]{Abbot}. Similarly, Bal defines the narrator as an "... agent cannot be identified with the writer, painter, or filmmaker... a fictitious spokesman, an agent technically known as the narrator."\cite[p. 8]{Bal}. Bal also goes on to note that the narrator does not relate continually; instead, the role of the narrator temporarily transfers to the speaking actor when direct speech occurs (a point Abbot agrees with).

The definition of what makes a narrator unreliable is a highly contested point. In her essay on fallible and untrustworthy narrators, Greta Olson first analyses Wayne Booth's introduction of the term unreliable and Ansgar N\"unning's criticism of Booth's definition. In addition, she offers an amplification of Booth's model of unreliability\cite{Olson}.