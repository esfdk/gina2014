\section{Results}
\label{Results}
We tested eight people during this project. It is a small sample size and it means that the conclusions we make based on these results are not documented very well. We would have liked to test thirty or more so that we had a more diverse sample size.

As our primary research goal was to find out how much influence the narrator has on player choice, our focus was on what the participants in our tests said and did in terms of following what the narrator told them to.
\begin{table}
\begin{center}
\renewcommand{\arraystretch}{1.3}
\caption {How many followed the narrators instructions}
\label{table-followed_instructions}
\begin{tabular}{ | p{2.3cm} | p{2.3cm} | p{2.3cm} | }
  \hline                       
  \textbf{Followed} & \textbf{Bit of both} & \textbf{Went against} \\
  \hline  
  3 & 2 & 3 \\
  \hline  
  For the reward & Wanted reward, but became suspicious that the narrator was lying & Wanted to see the narrator’s reaction \\
  \hline  
  Because there was nothing else to show the way & Ignored the narrator in the beginning, but wanted the larger reward at the end. & Felt that something shady was going on \\
  \hline  
\end{tabular}
\end{center}
\end{table}

As shown on table \ref{table-followed_instructions}, it is clear that the players went many different ways when they played the game. Based on the answers, it seems that the more the players felt the narrator was unreliable, the more willing they were to go any which way they wanted.

What is important to note is that all the players were influenced by the narrator in one way or another. None of them decided to go a certain way because they did not care about what the narrator told them to. 

While the players went different ways in the tests, only one of them felt that the decisions made during the game influenced the ending and outcome of the game. From the responses of the playtesters, it is clear that it was because there was nothing to compare their actual reward with whichever reward they believed they would get. There was no point in the game where they were told what they would receive at the end, so when they encountered a pile of coins in the final room, it was not clear if the size of the pile had increased or decreased as a result of their actions and choices in the game.


Another thing we felt was important to look at, was how good the players were at identifying the narrator as unreliable. The table below shows how many players that figured it out after the first and the second playthrough.
\begin{table}
\begin{center}
\renewcommand{\arraystretch}{1.3}
\caption {How many identified the narrator as unreliable and when}
\label{table-unreliable_narrator}
\begin{tabular}{ | p{2.3cm} | p{2.3cm} | p{2.3cm} | }
  \hline                       
  \textbf{First playthrough} & \textbf{Second playthrough} & \textbf{Never} \\
  \hline  
  3 & 8 & 0 \\
  \hline  
\end{tabular} 
\end{center}
\end{table}

As shown by table \ref{table-unreliable_narrator}, less than half the players figured out that the narrator was unreliable after the first playthrough, and the ones that did were not entirely sure about it. They had a suspicion that something the narrator said was incorrect, but they were could not point out why. After the second playthrough, all of them had figured it out, however.

The players who were suspicious during the first playthrough did not base their suspicion on contradictions in the game. Instead, they felt the narrator was unreliable because there was nothing to indicate that he was reliable. For them it was a lack of confirmation of reliability, rather than contradictions, that made them not trust him.
\subsection{The influence of the narrator}

%What can we use the results for?
%RQ 1:
%Narrator has influence in some way (at least in this particular case)
%If player believes the narrator to be reliable and/or trustworthy, they will follow directions given by the narrator, unless they are tempted to experiment
%If there is any slight mistrust or any belief that the narrator is unreliable, the player is much more likely to experiment

\subsection{The unreliability of the unreliable narrator}

%RQ 2:
%results indicate that some players do not need a strong contradiction to get the feeling that the narrator is unreliable, but instead a lack of confirmation that their actions matter is enough
%offering choices that should supposedly change the state and/or the outcome of the game makes it much easier to realise/identify a narrator as unreliable because multiple playthroughs makes it easily evident
