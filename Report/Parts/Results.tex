\section{Results}
\label{Results}

We had eight people test our game and answer questions.

We felt it was most important to look at whether the players followed our narrator or not. To summarize the relevant information, we created a table that shows how many players that did what, and why they made that choice. \\

\begin{tabular}{ | p{2.3cm} | p{2.3cm} | p{2.3cm} | }
  \hline                       
  \textbf{Followed} & \textbf{Bit of both} & \textbf{Went against} \\
  \hline  
  3 & 2 & 3 \\
  \hline  
  For the reward & Wanted reward, but became suspicious that the narrator was lying & Wanted to see the narrator’s reaction \\
  \hline  
  Because there was nothing else to show the way & Ignored the narrator in the beginning, but wanted the larger reward at the end. & Felt that something shady was going on \\
  \hline  
\end{tabular} \\

From the table, we can see that the players went many different ways when they played the game. Based on the answers, it seems that the more the players felt the narrator was unreliable, the more willing they were to go any which way they wanted.

What is important to note, is that all the players were influenced by the narrator in one way or another. None of them decided to go a certain way because they did not care about the narrator. 

While the players went different ways in the tests, none of them felt that their decisions had any influence on the reward at the end. This comes from the fact that they had nothing to compare the reward at the end with. At no point had they been told what they would receive for going the right way, so when they encountered a pile of money in the final room, it was unclear to them if the pile had increased or decreased through the game, based on their decisions. 

Another thing we felt was important to look at, was how good the players were at identifying the narrator as unreliable. The table below shows how many players that figured it out after the first and the second playthrough. \\

\begin{tabular}{ | p{2.3cm} | p{2.3cm} | p{2.3cm} | }
  \hline                       
  \textbf{First playthrough} & \textbf{Second playthrough} & \textbf{Never} \\
  \hline  
  3 & 8 & 0 \\
  \hline  
\end{tabular} \\

Less than half the players figured out that the narrator was unreliable after the first playthrough, and the ones that did were not entirely sure about it. They had a suspicion that something he said was wrong, but they were could not point out why. After the second playthrough, all of them had figured it out, however.

The players who were suspicious during the first playthrough did not base their suspicion on contradictions in the game. Instead, they felt the narrator was unreliable because there was nothing to indicate that he was reliable. For them it was a lack of confirmation of reliability, rather than contradictions to the opposite, that made them not trust him.


\subsection{The influence of the narrator}

%What can we use the results for?
%RQ 1:
%Narrator has influence in some way (at least in this particular case)
%If player believes the narrator to be reliable and/or trustworthy, they will follow directions given by the narrator, unless they are tempted to experiment
%If there is any slight mistrust or any belief that the narrator is unreliable, the player is much more likely to experiment

\subsection{The unreliability of the unreliable narrator}

%RQ 2:
%results indicate that some players do not need a strong contradiction to get the feeling that the narrator is unreliable, but instead a lack of confirmation that their actions matter is enough
%offering choices that should supposedly change the state and/or the outcome of the game makes it much easier to realise/identify a narrator as unreliable because multiple playthroughs makes it easily evident
