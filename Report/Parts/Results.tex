\section{Results}
The main results and findings go here.

%insert tables

%players did not feel that they had any influence on the outcome

%all players were influenced by the narrator

%if players see the narrator is reliable/trustworthy, they follow instructions, else they are much more likely to experiment

%when playing through a second time and making different choices, the players have a much easier time identifying the narrator is unreliable

%some players did not need contradictions to identify the narrator as unreliable, but simply a lack of confirmation made them suspicious

%the room was bland,boring and monotonous


\subsection{The influence of the narrator}

%What can we use the results for?
%RQ 1:
%Narrator has influence in some way (at least in this particular case)
%If player believes the narrator to be reliable and/or trustworthy, they will follow directions given by the narrator, unless they are tempted to experiment
%If there is any slight mistrust or any belief that the narrator is unreliable, the player is much more likely to experiment

\subsection{The unreliability of the unreliable narrator}

%RQ 2:
%results indicate that some players do not need a strong contradiction to get the feeling that the narrator is unreliable, but instead a lack of confirmation that their actions matter is enough
%offering choices that should supposedly change the state and/or the outcome of the game makes it much easier to realise/identify a narrator as unreliable because multiple playthroughs makes it easily evident
