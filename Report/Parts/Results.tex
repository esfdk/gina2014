\section{Results}
The main results and findings go here.

%Do not number an equation if it will not be directly cited in the
%report. In order to avoid numbered equations, use
%$\backslash$begin\{equation*\}--$\backslash$end\{equation*\},
%$\backslash$[ --$\backslash$], or \$\$--\$\$.  For example:
%\begin{equation*}
%a = b + c,
%\end{equation*}
%$$\dot x = f(x,u) + g(x,u),$$
%or
%\[\ddot s=G(s,t)\]
%where $f$, $g$, and $G$ are functions.
%
%Note that Equation (\ref{equation-eqn1}) below is numbered!  It is
%produced using $\backslash$begin\{equation\}--$\backslash$end\{equation\}:
%\begin{equation}
%F_i(P_i)=a_{i}+b_{i}P_i+c_{i}P_{i}^2
%\label{equation-eqn1}
%\end{equation}
%where $a_{i},\ b_{i}$, and $c_{i}$ are coefficients of unit $i$, and $P_i$
%represents some value for unit $i$.
%
%Aligning equations can be done with
%either the align or eqnarray commands.  Recently,
%$\backslash$begin\{align\}--$\backslash$end\{align\} has gained popularity
%over $\backslash$begin\{eqnarray\}--$\backslash$end\{eqnarray\}.
%
%Equation (\ref{equation-eqn2}) is produced using
%$\backslash$begin\{align\}--$\backslash$end\{align\}:
%\begin{align}
%\dot {x}_l=& \sum_{i = 1}^m {\frac{c_{P_{x_i} } e^{k_{x_i}\bar{x}_i} + c_{N_{x_i} }
%e^{ -  k_{x_i} \bar{x}_i}}{e^{k_{x_i} \bar{x}_i} + e^{ - k_{x_i} \bar{x}_i}}} \nonumber\\
%& + \frac{1}{2}\sum\limits_j^q (c_{P{u_j }} + c_{N _{u_j }} ) \nonumber\\
%y=& \ A_0 + A_1 \tanh (K_x \bar {x}) + B\tanh (K_u \bar {u}) \nonumber\\
% =& \ F(x),
%\label{equation-eqn2}
%\end{align}
%where $F(x)$ is a function.
%
%Equation (\ref{equation-eqn3}) represents the same equation produced
%using $\backslash$begin\{eqnarray\}--$\backslash$end\{eqnarray\}:
%\begin{eqnarray}
%\dot {x}_l&=& \sum_{i = 1}^m {\frac{c_{P_{x_i} } e^{k_{x_i}\bar{x}_i} + c_{N_{x_i} }
%e^{ - k_{x_i} \bar{x}_i}}{e^{k_{x_i} \bar{x}_i} + e^{ - k_{x_i} \bar{x}_i}}} \nonumber\\
%&&+ \frac{1}{2}\sum\limits_j^q (c_{P{u_j }} + c_{N _{u_j }} ) \nonumber\\
%y&=& \ A_0 + A_1 \tanh (K_x \bar {x}) + B\tanh (K_u \bar {u})\nonumber\\
%&=& \ F(x),
%\label{equation-eqn3}
%\end{eqnarray}
%where $F(x)$ is a function.  You get the idea!
%
%
%
%\subsection{Figures and Tables}
%Please follow the style in this sample paper when generating your figures
%and tables.
%
%\subsection{Page Limit and Overlength Page Charges}
%A paper submitted to this conference should be prepared in a
%single-spaced, two-column format.  Its length must be kept to 8
%pages or less.  In exceptional circumstances, up to two additional
%pages will be permitted for a charge of AUD\$100 per additional page.
%Table~\ref{table-tab1} shows the page limit and page charge schedule.
%
%% An example of a floating table.  Note that, the
%% \caption command should come BEFORE the table.  Table text will default
%% to \footnotesize.
%% The \label must come after \caption as always.
%\begin{table}
%\begin{center}
%%% Increase table row spacing; adjust to taste
%\renewcommand{\arraystretch}{1.3}
%\caption{Page Limit}
%\label{table-tab1}
%% The array package and the MDW tools package offers better commands
%% for making tables than plain LaTeX2e's tabular which is used here.
%\begin{tabular}{|c|c|}
%\hline
%Page limit: & 8\\
%\hline
%Excess page charge: & AUD\$100/page\\
%\hline
%\end{tabular}
%\end{center}
%\end{table}
%
%Another example of a table is shown in Table~\ref{table-tab2}.
%
%\begin{table}[h]
%\caption{A second table}
%\begin{center}
%\begin{tabular}{|c|c|c|c|c|c|}
%\hline
%\multicolumn{1}{|c|}{\raisebox{-1.50ex}[0cm][0cm]{\!Method\!}}
%& \multicolumn{1}{|c|}{Mean}
%& \multicolumn{1}{|c|}{Best}
%& \multicolumn{1}{|c|}{Mean}
%& \multicolumn{1}{|c|}{Maximum}
%& \multicolumn{1}{|c|}{Minimum} \\
%& time & time & cost & cost & cost\\ \hline
%A      &  $928.36$  &  $926.20$  &  $124793.5$ & $126902.9$ & $123488.3$ \\ \hline
%B      &  $646.16$  &  $644.28$  &  $124119.4$ & $127245.9$ & $122679.7$ \\ \hline
%C      &  $1056.8$  &  $1054.2$  &  $123489.7$ & $124356.5$ & $122647.6$ \\ \hline
%D      &  $632.67$  &  $630.36$  &  $123382.0$ & $125740.6$ & $122624.4$ \\ \hline
%\end{tabular}
%\label{table-tab2}
%\end{center}
%\end{table}
%
%Citations are included like so~\cite{book}.
%Multiple citations appear like this~\cite{conf,article}.
