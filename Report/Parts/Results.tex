\section{Results}
\label{Results}
We tested eight people during this project. It is a small sample size and it means that the conclusions we make based on these results are not documented very well. We would have liked to test thirty or more so that we had a more diverse sample size.

As our primary research goal was to find out how much influence the narrator has on player choice, our focus was on what the participants in our tests said and did in terms of following what the narrator told them to.
\begin{table}
\begin{center}
\renewcommand{\arraystretch}{1.3}
\caption {How many followed the narrators instructions}
\label{table-followed_instructions}
\begin{tabular}{ | p{2.3cm} | p{2.3cm} | p{2.3cm} | }
  \hline                       
  \textbf{Followed} & \textbf{Bit of both} & \textbf{Went against} \\
  \hline  
  3 & 2 & 3 \\
  \hline  
  For the reward & Wanted reward, but became suspicious that the narrator was lying & Wanted to see the narrator’s reaction \\
  \hline  
  Because there was nothing else to show the way & Ignored the narrator in the beginning, but wanted the larger reward at the end. & Felt that something shady was going on \\
  \hline  
\end{tabular}
\end{center}
\end{table}

As shown on table \ref{table-followed_instructions}, it is clear that the players went many different ways when they played the game. Based on the answers, it seems that the more the players felt the narrator was unreliable, the more willing they were to go any which way they wanted.

What is important to note is that all the players were influenced by the narrator in one way or another. None of them decided to go a certain way because they did not care about what the narrator told them to. 

While the players went different ways in the tests, only one of them felt that the decisions made during the game influenced the ending and outcome of the game. From the responses of the playtesters, it is clear that it was because there was nothing to compare their actual reward with whichever reward they believed they would get. There was no point in the game where they were told what they would receive at the end, so when they encountered a pile of coins in the final room, it was not clear if the size of the pile had increased or decreased as a result of their actions and choices in the game.


Another thing we felt was important to look at, was how good the players were at identifying the narrator as unreliable. The table below shows how many players that figured it out after the first and the second playthrough.
\begin{table}
\begin{center}
\renewcommand{\arraystretch}{1.3}
\caption {How many identified the narrator as unreliable and when}
\label{table-unreliable_narrator}
\begin{tabular}{ | p{2.3cm} | p{2.3cm} | p{2.3cm} | }
  \hline                       
  \textbf{First playthrough} & \textbf{Second playthrough} & \textbf{Never} \\
  \hline  
  3 & 8 & 0 \\
  \hline  
\end{tabular} 
\end{center}
\end{table}

As shown by table \ref{table-unreliable_narrator}, less than half the players figured out that the narrator was unreliable after the first playthrough, and the ones that did were not entirely sure about it. They had a suspicion that something the narrator said was incorrect, but they were could not point out why. After the second playthrough, all of them had figured it out, however.

The players who were suspicious during the first playthrough did not base their suspicion on contradictions in the game. Instead, they felt the narrator was unreliable because there was nothing to indicate that he was reliable. For them it was a lack of confirmation of reliability, rather than contradictions, that made them not trust him.
\subsection{The influence of the narrator}
As our results show, there is an influence on the player by the narrator. While it is natural to assume that some non-story-based games will have much less influence (perhaps none at all), most games with an amply sized narrative will see the narrator influence player decisions. 

The degree to which the narrator is influential appears to be based on how much focus the player puts on what the narrator is relating to the player. If the player does not care what the narrator is saying, then the narrator will have almost no influence compared to cases where a player is paying a lot of attention to what is being narrated.

Additionally, our results indicate that if the player believes the narrator to be reliable and/or trustworthy, then the player is more likely to follow instructions or directions given by the narrator. In contrast, if the player has any mistrust in the narrator, then the player is much more likely to experiment or go his own way.

These points leads us to suggest that game narrative/story writers take great care in how they present the narrator to the player as it can make a big difference in how the player makes decisions in the game. Additionally, we suggest that authors use a narrator that relates personally to the player (and/or player character) and/or use a narrator that is close to the narrative in order to have the narrator be influential on the the decision-making of the player.
\subsection{The unreliability of the unreliable narrator}
Our results indicate that players do not necessarily need a contradiction in order for them to doubt the reliability of the narrator. In our test game, the narrator is not contradicted at all (until the last room, where a tiny hint is given), but even so, a few players felt that because they were not shown that the narrator was reliable, he was instead unreliable. However, it was not until the second playthrough that people gave clear indication that they believed the narrator to be untrustworthy (and not just fallible).

Considering the test results, we believe the answer to our second research question is that the player can experience a narrator as, at least, unreliable without being shown a contradiction to any statement made by the narrator.

Based on what we saw in the results, we would suggest that further experimentation is done in relation to only showing the narrator being contradicted across multiple playthroughs. If the player believes that during the first playthrough the narrative is being related to him/her by a reliable source, it would be very interesting to observe the reactions of the player if during a second playthrough, he/she is given a clear contradiction in the narrative to make them doubt the reliability of the narrator.