\section{Conclusions}
In this report, we have covered what constitutes a narrator and can make a narrator unreliable. We have also offered two hypothesises on how influential the narrator can be on the choices that players make in games. Our hypothesis are that the stronger the personal relation is between the narrator and the player and/or player character, the more influential the narrator is. Additionally, if the narrator is close to the narrative and the player and/or player character is strongly opinionated about the narrator, the narrator will have greater influence on player choice.

We use a simple game made in Unity to test our two research questions; 
\\(1) \textit{"What influence does the narrator have on decision making in games?"}, 
\\(2) \textit{"Does the player experience an untrustworthy narrator as reliable if almost no contradictions are given in the narrative?"}
\\The game consists of a series of rooms where the player has the choice of two doors. A distant third-person, untrustworthy narrator that is externally focalized on the player character to relate the narrative to the player.

Additionally, we have described the questions and the testing method we used to gather results we used to analyse and answer our research questions.

The results of our tests showed that players are highly influenced by what the narrator is relating to them and will in most cases either follow the instructions given by the narrator or try to do the opposite of what the narrator is asking in order to experiment with the game. We also conclude that while not all players will notice an unreliable narrator on the first playthrough, the possibility of playing through a game multiple times and making different choices in said playthroughs open up for the possibility of making identifying the narrator as unreliable only possible if the game is played more than once.

These conclusions leads us to suggest that game narrative and story writers keep in mind that if they want an influential narrator, they must make the player strongly opinionated about the narrator and/or make the relation with the narrator personal to the player and/or the player character.

\subsection{Improving the project}
There are many elements in the project that could be improved to increase the quality of the project if further work is done. We have listed them in the order of importance.

The number of people we tested was far too few to form a solid conclusion based on the feedback we received. Testing a minimum of thirty people would ensure that there should at least be a third of our testers that took the same path. Players might take the same path for different reasons, which is why this would be interesting. In addition, most of our testers were also used to playing games, so the feedback is biased towards more experienced gamers. This could possibly have the effect that the testers are more likely to try to "game" by testing the boundaries of what is possible\footnote{One of our testers actually did so - he spent a minute or two in one of the test rooms trying to find a different way out.}, instead of making one of the two choices given to them by the narrator. This is also more likely to happen because the test happens in a 3D-game environment instead of say a test-based environment.

The voice acting of the narrator could be improved significantly. We did it internally on the team and because we do not have great voice acting capabilities and below average quality recording equipment, we end up with poor quality in voice acting. An easy and cheap method of improving this would be to have subtitles available, although that could possibly detract from the tone of the narrator's voice.

Creating more varied game play mechanics would make the game have more variance and would likely make playing through a longer game less dull. This would lead to the opportunity of testing a more varied set of combinations in player choice and thus a better indication of the narrator's influence on the player.

The rooms in the game are completely bare and grey, which can lead to a monotonous player experience. Furnishing the rooms with simple furniture would not be very distracting from the experience and would make it so that the narrator is not the entire focus of the experience.

Using some of the research that Porteous et al. \cite{Por} did on character's point of view and narrative actions, it could be possible to observe how players react to the same given choice from different perspectives, e.g. a greedy or self-serving researcher attempting to direct the current protagonist to make the wrong choices to save money on his research budget.

\subsection{Who did what?}
Jakob Melnyk wrote the dialogue and did the voice acting. Jacob Grooss programmed the game and made the scenes in Unity. Equal work was done on writing the report.