\section{Conclusions}

\subsection{Improving the project}
There are a few major points on the project that should certainly be improved if further work is done. 

Creating more varied game play mechanics would make the game have more variance and would likely make playing through a longer game less dull. This would lead to the opportunity of testing a more varied set of combinations in player choice and thus a better indication of the narrator's influence on the player.

The voice acting of the narrator could be improved significantly. We did it internally on the team and because we do not have great voice acting capabilities and below average quality recording equipment, we end up with poor quality in voice acting. An easy and cheap method of improving this would be to have subtitles available, although that could possibly detract from the tone of the narrator's voice.

The rooms in the game are completely bare and grey, which can lead to a monotonous player experience. Furnishing the rooms with simple furniture would not be very distracting from the experience and would make it so that the narrator is not the entire focus of the experience.

Using some of the research that Porteous et al. \cite{Por} did on character's point of view and narrative actions, it could be possible to observe how players react to the same given choice from different perspectives, e.g. a greedy or self-serving researcher attempting to direct the current protagonist to make the wrong choices to save money on his research budget.

\subsection{Who did what?}
Jakob Melnyk wrote the dialogue and did the voice acting. Jacob Grooss programmed the game and made the scenes in Unity. Equal work was done on writing the report.