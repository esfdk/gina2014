\section{Introduction}
\label{Introduction}

This report has been written as part of a project in the Interactive Narrative course on the Master line of Games at the IT-University of Copenhagen.\\

In this report, we will start by looking at what Greta Olson says about fallible and untrustworthy narrators compared to what Bal says about unreliable narrators. We will use this information to form a view on what unreliable narrators are, and what effect they can have in a narrative.

After determining what an unreliable narrator is, we will look at what we want to research in the project: What influence, if any, a narrator has on a player's decision-making in games and if a player can determine if a narrator is unreliable, given that there is only the player and the narrator. We will discuss these questions, why we want to look into them and what we hope to learn.

Next, we will talk about the method we will use to gather information to answer the questions, which is a simple game made to test the influence of a narrator on the player. We will discuss how we made the game and our reasoning behind the different choices we made with regards to the game. We will also discuss our narrator and why we created different audio for every part of the game.

After the method, we will discuss the results we got TODO: Get results and talk about them here.

Lastly, we conclude whether we manged to answer our research questions or not and how we could improve on the project in the future.

%\PARstart{I}{f} you have an introduction for your paper, put it
%here.
%
%This sample file is intended to serve as a ``starter file."  You
%need to replace the text in this file with the text that makes up
%your paper.
%
%\subsection{Subsection Heading Here}
%If applicable, subsection text goes here.
%
%\subsubsection{Subsubsection Heading}
%Insert any subsubsection text here.  Same thing as before --- you may
%or may not have any subsubsections.
%
%\subsubsection{About This Template}
%This template is for LaTeX users of the Advanced AI in games class.
%Authors should use this sample paper as a guide in the production of
%their report(s).
