\section{Introduction}
\label{Introduction}

This report has been written as part of a project in the Interactive Narrative course on the Master of Games at the IT-University of Copenhagen.\\

What effect does a narrator having on player choice in games? In this report, we will attempt to answer this question by reviewing the different definitions of a narrator presented by Mieke Bal and H. Porter Abbot and the different definitions by Wayne Booth, Ansgar N\"unning and Greta Olson of what makes a narrator unreliable in regards to the narrative that is being related. We will use the different definitions to define our use of the concept of an unreliable narrator and then address what effect they can have in an interactive narrative. After reviewing the theory behind what defines a narrator and what makes a narrator unreliable, we will present two hypothesises regarding the influence of a narrator on player choice.

Following the presentation of our hypothesises, we will discuss the motivation behind the research questions presented in this report: What influence, if any, a narrator has on a player's decision-making in games and if a player can determine if a narrator is unreliable, given that there is only the player and the narrator in the narrative. 

Next, we will discuss the method we used to gather the data we use to conclude on our research questions - a simple game made to test the influence that a narrator has on the player. We will discuss the thought process behind the concept of the game and the design choices we made in terms of length and variance in the game. In addition, we will explain the gameplay, the structure of the game, how the narrator acts in the game and why we chose the game length that we did. Then we will list the questions that we asked the players that tested our game and the motivation behind them.

Additionally, we will present and analyse the results of our testing. Our results indicate that narrators influence every decision made by the players in either a positive or negative way. The results also showed that being also to experience the narrative of the game in two different ways made it much easier to identify the narrator as being unreliable.

Finally, we conclude that the narrator can influence the players decision-making greatly depending on his position in the narrative and so we suggest that authors of narrative in interactive fiction strongly consider how much they want the narrator (and the narration) to influence the decisions made by the player. We also conclude that because it is much easier to identify the narrator as unreliable in subsequent playthroughs, it is not necessary to make the contradictions of the narrator obvious in a single playthrough, but instead it can span multiple different playthroughs.