\subsection{Influence of the narrator in interactive fiction}
In games where the narrator is relating a narrative to the player and that narrative in itself can vary based on player choices, it is trivially obvious that the narrator has some level of influence on the decisions that the player makes. It is not unreasonable to assume that the level of influence heavily depends on who is doing the narration (the voice\cite[p. 70]{Abbot}) and how entwined in the plot the narrator is (the distance \cite[p. 74]{Abbot}). As we have not found any research or related work on this particular subject, we will try to present our own hypothesis' on the subject.

If the player character is narrating in the first person or the narrator speaking in the third voice while "looking through the eyes" (thus the protagonist is the focalizer\cite[p. 73]{Abbot}), it is likely that the narrator's level of influence is high as there is a kind of personal relation with the player (and the player character). This personal relation attempts to influence the player by making the player feel like the narrator is actually relating what the character itself would do (even though this is not necessarily always the case). If, instead, the narrator speaks with a third-person voice and utilises an external\footnote{The focalizer looks at the character from the outside and does not know the internal workings of the character.} or no (zero\footnote{The focalizer is omniscient and is not bound to a specific character}) focalizer, the narrative seems much less personal and we reason that a less personal narrative will influence player decisions less.

Another hypothesis of ours is that if a character that is also the narrator is close to the narrative, their influence on player decision is dependent on their likeability. If the player and/or player character has a neutral opinion of the character, they are less likely to have an influence on the decisions of the player compared to a character that the player and/or player character has strong opinion about. In the case of the neutral opinion, the player is likely to not care what the narrator is attempting to convince the player to do. If the opinion of a character is strong, the player is likely to either do what the narrator is asking the player to do (if the opinion is positive) or do the opposite of what the narrator is saying (if the opinion is negative). If the narrator is distant from the narrative, then he is neutral in terms of influence on the player's decisions.