\subsection{Unreliable Narrators}
In this paper, we define a narrator using Abbots definition; "One who tells a story"\cite[p. 238]{Abbot}. A narrator is not the same as the author of a work. Instead the author of the work uses the narrator as a device to present the narrative. Many narrative works use a single unidentified\footnote{Unidentified as in the narrator being unknown to the narratee, which may make narratee confuse author and narrator.} narrator, often in combination with third-person narration. Other narrative works use multiple narrators and/or identified narrators\footnote{An identified narrator is a narrator that is given name, identity and/or place in the narrative in some sense, e.g. a character in the story world describing what happened to him the day before.} to tell the narrative, which opens up for the possibility of unreliability in the narrators.

According to Olson, Booth defines unreliable narrators as narrators "who articulate values and perceptions that differ from those of the implied author."\cite[p. 94]{Olson}