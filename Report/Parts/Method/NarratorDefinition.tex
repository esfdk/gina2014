\subsection{What is a narrator?}
There are many definitions of what a narrator is. While many of these definitions are similar, they each have their own different nuances. Bal defines the narrator as an "... agent cannot be identified with the writer, painter, or filmmaker... a fictitious spokesman, an agent technically known as the narrator."\cite[p. 8]{Bal}. He also notes that the role of narrator does not belong to a single agent. Instead, the role of the narrator belongs to the agent that is currently relating the story to the narratee\footnote{The narratee is the receiver of the narrative. The narratee can be both a fictitious character and/or the reader.}. 

Abbot's definition of a narrator being "One who tells a story"\cite[p. 238]{Abbot} is simpler, yet somewhat similar. He does not agree with Bal that the narrator cannot be identified with the author of the work. In a counter-argument made to Barthes theory that "The living author of a narrative is in no way to be confused with the narrator of that narrative" \cite[p. 261]{Barthes}, Abbot notes that if narrator is telling the story of the author in the voice of the author, can he be entirely separate? The answer is not simple and beyond the scope of this paper, but suffice it to say that it is only in rare circumstances the point becomes relevant. With the exception of the autobiographical narrative, Abbot agrees that the narrator is not the same as the author of a work. Instead, Abbot states that the narrator is a device used in combination with other devices to construct the narrative told to the reader and/or narratee \cite[p. 69]{Abbot}.

In many narratives, the author uses a single, unidentified\footnote{Unidentified as in the narrator being unknown to the narratee, which may make narratee confuse author and narrator.} narrator to relate the narrative to the receivers. Depending on the focalization of the narrator and the narrator's distance from the narrative, the narrator may not be part of the narrative at all (except for relating the narrative itself). On the other side of the coin, a narrator may be part of the narrative itself, but then he, she or it will often be an identified\footnote{An identified narrator is a narrator that is given name, identity and/or place in the narrative in some sense, e.g. a character in the story world describing what happened to him the day before.} narrator. If the narrative is being related "as it happens"\footnote{Most narrative works related in the present tense are not related as they happen, but instead the narrator relates them as if they were happening at the time, e.g. films and theatre plays} by an identified narrator, said narrator is often part of the narrative in the sense that they are part of the story being narrated.