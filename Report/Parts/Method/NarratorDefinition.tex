\subsection{What is a narrator?}
There are many definitions of what a narrator is. While many of these definitions are similar, they each have their own different nuances. Bal defines the narrator as an "... agent cannot be identified with the writer, painter, or filmmaker... a fictitious spokesman, an agent technically known as the narrator."\cite[p. 8]{Bal}. He also notes that the role of narrator does not belong to a single agent. Instead, the role of the narrator belongs to the agent that is currently relating the story to the narratee\footnote{The narratee is the receiver of the narrative. The narratee can be both a fictitious character and/or the reader.}. 

Abbot's definition of a narrator being "One who tells a story"\cite[p. 238]{Abbot} is simpler, yet somewhat similar. He does not agree with Bal that the narrator cannot be identified with the author of the work. In a counter-argument made to Barthes theory that "The living author of a narrative is in no way to be confused with the narrator of that narrative" \cite[p. 261]{Barthes}, Abbot notes that if narrator is telling the story of the author in the voice of the author, can he be entirely separate?

In this paper, we define a narrator using Abbots definition; "One who tells a story"\cite[p. 238]{Abbot}. A narrator is not the same as the author of a work. Instead the author of the work uses the narrator as a device to present the narrative. Many narrative works use a single unidentified\footnote{Unidentified as in the narrator being unknown to the narratee, which may make narratee confuse author and narrator.} narrator, often in combination with third-person narration. Other narrative works use multiple narrators and/or identified narrators\footnote{An identified narrator is a narrator that is given name, identity and/or place in the narrative in some sense, e.g. a character in the story world describing what happened to him the day before.} to tell the narrative, which opens up for the possibility of unreliability in the narrators.