\subsection{Unreliable Narrators}
Because a narrator can be anyone telling a story, it is inevitable that the reliability of a narrator comes into question. If a narrator is close to the narrative; has a personal agenda in conflict with the truth\footnote{An example could be that the narrator is an accused at a trial, who is lying to escape punishment.}; and/or is insane, he/she might either lie about has actually happened or may even be wrong. 

Booth understands the concept of narrator unreliability to be a function of irony. He describes the speaker, or narrator, as being the butt of the ironic point. "The author and reader are secretly in collusion, behind the speaker's back, agreeing upon the standard by which he is found wanting" \cite[p. 94]{Olson}. In order for this collusion to take place, the reader must locate a sensibility behind the narrative that accounts for how it is constructed. This sensibility is what Booth calls an "implied author" \cite[p. 94]{Olson} and what Abbot calls an "inferred author"\cite[p. 84-85]{Abbot}.

Abbot suggests the use of the "inferred" author because the sensibility, that we, as interpreters, use to account for the narrative, can vary from interpreter to interpreter. This can lead to interpreters giving values to the implied author that the original real author did not intend for the implied author to have. This point in particular has been some of the ammunition used in attacks on the concept of the implied author.\cite[p.85]{Abbot}

\insertPicture{0.45}{Model_Booth}{Booth's model for unreliable narrative \cite[p. 99]{Olson}}

Booth's model (see figure \ref{fig:Model_Booth}, page \pageref{fig:Model_Booth}) for identifying an unreliable narrator is simple: one simply reads (or listens) until markers in the work force one to revise one's interpretation. According to Booth, an unreliable narrative is a textual constant because an objective standpoint exists from which insightful readers will judge irony and reliability.\cite[p. 95]{Olson}

Ansgar N\"unning is one of the critics of the implied author. N\"unning believes that critics treat the narrator as a real person whose lack of reliability they perceive as a failure or a limitation. This leads to unreliable narrators with easily identifiable moral failings being overemphasised. Additionally, the use of the implied author can lead interpreters to projecting their own general values onto the implied author. This projection can lead to the conclusion that a narrator is unreliable because of their morally unacceptable philosophy - a judgement based on their own moral philosophy and not necessarily the one belonging to the author. \cite[p. 96-97]{Olson} 

\insertPicture{0.45}{Model_Nunning}{N\"unning's model for unreliable narrative \cite[p. 99]{Olson}}

N\"unnings model (see figure \ref{fig:Model_Nunning}, page \pageref{fig:Model_Nunning}) differs from Booth's primarily on the fact that N\"unning recognises the fact that every reading of work is limited and situational and that every reader is potentially unreliable\cite[98]{Olson}. In both models three points of views coexist: the personified narrator, the implied author/totally of textual signals, and the reader who is trying to make sense of the narrator and the implied author. It is the case in both models that a narrator can only be identified as unreliable if the reader perceives a divergence of value, opinion, etc. between the narrator and the implied author/totality of textual signals.

The main difference between Booth's and N\"unning's models is where the authority to judge as unreliable lies: Booth's model gives the authority to an implied author whose norms form the basis for judging unreliability, while N\"unning's model makes use the of the limited validity of subjective reader response.

Amplifying Booth's model, Olson suggests a definition of unreliable narrators as the following \cite[101-102]{Olson}:
\begin{itemize}
	\item \textbf{Reliable narrators} report with factual accuracy and are in accord with the values of their narratives.
	\item \textbf{Fallible narrators} do not reliably report on narrative events because they are mistaken about their judgement, perceptions or are biased\footnote{The perceptions of fallible narrators can be impaired due to a number of reasons, e.g. limited experience, the source of their information biased and/or incomplete}.
	\item \textbf{Untrustworthy narrators} are dispositionally unreliable. The inconsistencies they display can be due to either ingrained behavioural traits or some current self-interest. Untrustworthiness is a distinct characteristic of the narrator.
\end{itemize}

We use Olson's amplification of Booth's model to differentiate the different types of unreliable narrators in this paper.