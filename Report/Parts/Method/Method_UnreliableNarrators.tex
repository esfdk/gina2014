\subsection{Unreliable Narrators}
Because a narrator can be anyone telling a story, it is inevitable that the reliability of a narrator comes into question. If a narrator is close to the narrative, has a personal agenda in conflict with truth\footnote{An example could be that the narrator is an accused at a trial, who is lying to escape punishment.} and/or is insane, he/she might either lie about has actually happened or may even be wrong. 

Booth understands the concept of narrator unreliability to be a function of irony. He describes the speaker, or narrator, as being the butt of the ironic point. "The author and reader are secretly in collusion, behind the speaker's back, agreeing upon the standard by which he is found wanting" \cite[p. 94]{Olson}. In order for this collusion to take place, the reader must locate a sensibility behind the narrative that accounts for how it is constructed. This sensibility is what Booth calls an "implied author" \cite[p. 94]{Olson} and what Abbot calls an "inferred author"\cite[p. 84-85]{Abbot}.

Abbot suggests the use of the "inferred" author because the sensibility, that we, as interpreters, use to account for the narrative, can vary from interpreter to interpreter. This can lead to interpreters giving values to the implied author that the original real author did not intend for the implied author to have. This point in particular has been some of the ammunition used in attacks on the concept of the implied author.\cite[p.85]{Abbot}

Booth's model for identifying an unreliable narrator is simple: one simply reads (or listens) until markers in the work force one to revise one's interpretation. According to Booth, an unreliable narrative is a textual constant because an objective standpoint exists from which insightful readers will judge irony and reliability.\cite[p. 95]{Olson}

Ansgar N\"unning is one of the critics of the implied author. N\"unning believes that critics treat the narrator as a real person whose lack of reliability they perceive as a failure or a limitation. This leads to unreliable narrators with easily identifiable moral failings being overemphasised. Additionally, the use of the implied author can lead interpreters to projecting their own general values onto the implied author. This projection can lead to the conclusion that a narrator is unreliable because of their morally unacceptable philosophy - a judgement based on their own moral philosophy and not necessarily the one belonging to the author. \cite[p. 96-97]{Olson} 



Olson's definition of unreliable narrators are\cite[p. 96]{Olson}

\begin{itemize}
	\item \textbf{Reliable narrators} speaks or acts in accordance with the norms , while \textbf{Unreliable narrators} do not.
	\item \textbf{Untrustworthy narrators} relate information that they know to be false, often relating a narrative beneficial to a person agenda.
	\item \textbf{Inconscient (unconscious narrators} believe themselves to be correct or believe themselves to have qualities that they do not possess - these types of narrators can also be referred to as \textbf{fallible narrators}.
\end{itemize}