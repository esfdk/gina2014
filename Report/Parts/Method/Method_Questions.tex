\subsection{Test Questions}
\label{Method_Questions}
To gather results, we used the "observe and interview" method. We observed the decisions that the players made in terms of which path they take through the game. Following the observation, we interview the players and ask them the questions presented below. In an effort not to influence the responses of the players, we have made the questions as objective and neutral as we possibly could. Additionally, we have attempted to not present the player with "yes/no" questions so that they are forced to explain their answer. In order to make it easier for the player to recognise the narrator as being untrustworthy, we have them play through the game once again (hopefully taking a different path) after they answer the first six questions.

\begin{itemize}

	\item \textit{"Why did you take the path you took?"} \\
 		This question would give us a clear idea of why the player made the decisions they made. Was it because of the reward the narrator talked about, was it just on impulse or was it to try to trick the narrator? \\

	\item \textit{"What did you think about the narrator?"} \\
		 This question gives us an indication of the how the player experienced the narrator; specifically how useful, reliable and trustworthy the narrator was. \\

	\item \textit{"In what way, if any, did the narrator influence the decisions you made in the game?"} \\
		 Because of our first research question, we are very interested in the influence the player felt from the narrator. This question might be partially answered by the first question, but we ask this question to make sure that we have a clear indication of the narrator's influence. \\

	\item \textit{"Would you do anything differently if you were to play again?"} \\
		 If the player is not interested in doing anything differently, it might be because the narrator had a lot of influence or none at all or anything in between. Depending on their path taken and their answer, it might be necessary to dig deeper into this answer. \\
	\item \textit{"What did you think about the different rooms you went through?"} \\
		 This question was for figuring if the player had even thought about the rooms or not, and if yes whether the rooms had any influence on the decisions of the player or not. \\

	\item \textit{"What influence did you feel the choices you made had on the outcome?"} \\
		 This question would give us an idea about whether the player felt that the narrator had been unreliable or not through the rooms. If the player felt their choices had no influence, we could deduce that the narrator had felt unreliable. \\

	\item \textit{"In what way, if any, did your opinion of the narrator change after the second play through?} \\
		  The motivation behind this question is to check if the player notices that the ending does not change even though they take a different path through the game.
\end{itemize}

We chose to not have more questions, as we felt these were enough to gather the information we were interested in. More questions might have yielded more information, but it would have been redundant as it would either cover what these questions cover, or be of no interest with regards to our research questions.

TODO: HOW TO TEST!?!!?!!?!!ONEONEONE