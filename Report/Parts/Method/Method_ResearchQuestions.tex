\subsection{Our research questions}
\label{Method_Research}

In this paper, we attempt to answer two different research questions.\\

\textbf{Research question 1:}
\begin{center}
\textit{"What influence does the narrator have on decision making in games?"}
\end{center}
Many AAA-games (and smaller productions as well) contain a lot of narrative focused on telling the story in the games story world. Most of these games do not attempt to influence the player's actions using the general overall narrator used in most games. Instead they use the game character's as narrators during direct speech to guide the player through the (illusions of) choices presented to them in the game. A few games have attempted to influence the player through the general narrator, either by stating what the player is doing in the game\footnote{Bastion is a popular example of this.} or by breaking the fourth wall\footnote{Stanley Parable is one of the best examples of this in games. The narrator tells the character what to do; if ignored, he speaks directly to the player and gets angry that the player is not following his story.}. Our intention is to find out if the player makes the choices that the narrator relates as beneficial to the character that the player is playing or if the curiosity of the player makes the player test the narrative.

In order for us to answer this question, we have made a very short game in Unity with simple assets and mechanics. The game uses a distant third-person, untrustworthy narrator that is externally focalized on the player character to relate the narrative to the player. The player then makes decision as to which path to take through the game while the narrator is trying to influence the player into taking a specific path.

\textbf{Research question 2:}
\begin{center}
\textit{"Does the player experience an untrustworthy narrator as reliable if almost no contradictions are given in the narrative?"}
\end{center}
As we were making the game, we looked at other work relating to unreliable narrators and found Olson's "Reconsidering Unreliability: Fallible and Untrustworthy Narrators" in which she states that an unreliable narrator can only be identified as unreliable if the narrator announces that he/she is insane or if the reader is given evidence that contradicts what the narrator is saying\cite[p. 104]{Olson}. This point is interesting to us, because in the test game we made, we have tried to make sure that there is very little evidence of the fact that the narrator is unreliable and untrustworthy. It becomes much more evident if the game is played through more than once, but in a single play through there is only a slight hint that the narrator may not be related the whole truth\footnote{A more philosophical approach could be to question whether the narrator is even unreliable if there are no contradictions in the narrative.}. As such, we are interested in figuring out if the players are able to identify the narrator as unreliable by having them play through the game more than once and having them make different choices.