\documentclass[conference]{IEEEtran}
% If the IEEEtran.cls has not been installed into the LaTeX system files,
% manually specify the path to it.  e.g.
% \documentclass[conference]{./IEEEtran}

% Add and required packages here
\usepackage{graphicx,times,amsmath}

% Correct bad hyphenation here
\hyphenation{op-tical net-works semi-conduc-tor IEEEtran}

% To create the author's affliation portion using \thanks
\IEEEoverridecommandlockouts

\textwidth 178mm
\textheight 239mm
\oddsidemargin -7mm
\evensidemargin -7mm
\topmargin -6mm
\columnsep 5mm

\begin{document}

% Project title: keep the \ \\ \LARGE\bf in it to leave enough margin.
\title{\ \\ \LARGE\bf Title of your Report}

\author{Jacob Claudius Grooss, Jakob Melnyk}

% Uncomment out the following line for invited papers
%\specialpapernotice{(Invited Paper)}

% Make the title area
\maketitle

\section{Introduction}
\label{Introduction}

This report has been written as part of a project in the Interactive Narrative course on the Master of Games at the IT-University of Copenhagen.\\

What effect does a narrator having on player choice in games? In this report, we will attempt to answer this question by reviewing the different definitions of a narrator presented by Mieke Bal and H. Porter Abbot and the different definitions by Wayne Booth, Ansgar N\"unning and Greta Olson of what makes a narrator unreliable in regards to the narrative that is being related. We will use the different definitions to define our use of the concept of an unreliable narrator and then address what effect they can have in an interactive narrative. After reviewing the theory behind what defines a narrator and what makes a narrator unreliable, we will present two hypothesises regarding the influence of a narrator on player choice.

Following the presentation of our hypothesises, we will discuss the motivation behind the research questions presented in this report: What influence, if any, a narrator has on a player's decision-making in games and if a player can determine if a narrator is unreliable, given that there is only the player and the narrator in the narrative. 

Next, we will discuss the method we used to gather the data we use to conclude on our research questions - a simple game made to test the influence that a narrator has on the player. We will discuss the thought process behind the concept of the game and the design choices we made in terms of length and variance in the game. In addition, we will explain the gameplay, the structure of the game, how the narrator acts in the game and why we chose the game length that we did. Then we will list the questions that we asked the players that tested our game and the motivation behind them.

Additionally, we will present and analyse the results of our testing. Our results indicate that narrators influenced every decision made by the players in either a positive or negative way. The results also showed that being also to experience the narrative of the game in two different ways made it much easier to identify the narrator as being unreliable.

Finally, we conclude that the narrator can influence the players decision-making greatly depending on his position in the narrative and so we suggest that authors of narrative in interactive fiction strongly consider how much they want the narrator (and the narration) to influence the decisions made by the player. We also conclude that because it is much easier to identify the narrator as unreliable in subsequent playthroughs, it is not necessary to make the contradictions of the narrator obvious in a single playthrough, but instead it can span multiple different playthroughs.

\section{Related Work}
Here you describe what other work has been done on this topic and closely related topics. It is important that how your work differs from previous work.  


\section{Method}
\label{Method}

\subsection{What is a narrator?}
There are many definitions of what a narrator is. While many of these definitions are similar, they each have their own different nuances. Bal defines the narrator as an "... agent cannot be identified with the writer, painter, or filmmaker... a fictitious spokesman, an agent technically known as the narrator."\cite[p. 8]{Bal}. He also notes that the role of narrator does not belong to a single agent. Instead, the role of the narrator belongs to the agent that is currently relating the story to the narratee\footnote{The narratee is the receiver of the narrative. The narratee can be both a fictitious character and/or the reader.}. 

Abbot's definition of a narrator being "One who tells a story"\cite[p. 238]{Abbot} is simpler, yet somewhat similar. He does not agree with Bal that the narrator cannot be identified with the author of the work. In a counter-argument made to Barthes theory that "The living author of a narrative is in no way to be confused with the narrator of that narrative" \cite[p. 261]{Barthes}, Abbot notes that if narrator is telling the story of the author in the voice of the author, can he be entirely separate? The answer is not simple and beyond the scope of this paper, but suffice it to say that it is only in rare circumstances the point becomes relevant. With the exception of the autobiographical narrative, Abbot agrees that the narrator is not the same as the author of a work. Instead, Abbot states that the narrator is a device used in combination with other devices to construct the narrative told to the reader and/or narratee \cite[p. 69]{Abbot}.

In many narratives, the author uses a single, unidentified\footnote{Unidentified as in the narrator being unknown to the narratee, which may make narratee confuse author and narrator.} narrator to relate the narrative to the receivers. Depending on the focalization of the narrator and the narrator's distance from the narrative, the narrator may not be part of the narrative at all (except for relating the narrative itself). On the other side of the coin, a narrator may be part of the narrative itself, but then he, she or it will often be an identified\footnote{An identified narrator is a narrator that is given name, identity and/or place in the narrative in some sense, e.g. a character in the story world describing what happened to him the day before.} narrator. If the narrative is being related "as it happens"\footnote{Most narrative works related in the present tense are not related as they happen, but instead the narrator relates them as if they were happening at the time, e.g. films and theatre plays} by an identified narrator, said narrator is often part of the narrative in the sense that they are part of the story being narrated.

\subsection{Unreliable Narrators}
According to Olson, Booth defines unreliable narrators as narrators "who articulate values and perceptions that differ from those of the implied author."\cite[p. 94]{Olson} 

\subsection{Research questions}
In this paper, we attempt to answer two different research questions.

Research question 1:
\begin{center}
\textit{"What influence does the narrator have on decision making in games?"}
\end{center}
Our interest in this question is to learn whether people simply do what the narrator says, if they actively oppose what the narrator is saying or if they ignore the narrator.

Research question 2:
\begin{center}
\textit{"Does the player experience an unreliable narrator as reliable if no contradictions are given in the narrative?"}
\end{center}

In order for us to answer our research questions, we made a very short game in Unity with simple assets and mechanics. We had people playtest the game and ask them questions pertaining to our research questions.

\subsection{The Game}
\label{Method_Game}

The game we created requires the player to go through four rooms, each room presenting the player with a choice between two doors; one on the right side of the room and one on the left side of the room. After the four rooms, the player ends in a final room with no exits and the game is over. The structure of the rooms can be seen in figure \ref{fig:RoomStructure} on page \pageref{fig:RoomStructure}.

\insertPicture{0.45}{RoomStructure}{The structure of the rooms the player goes through.}

The first four rooms the player goes through are identical, with an entrance and two exits. The room is square and the walls are grey (see figure \ref{fig:Screenshot_FirstRoom} on page \pageref{fig:Screenshot_FirstRoom} for a picture). There are no objects in the rooms, as we only wanted the narrator to be influencing the player's decision. Had we placed anything else in the rooms, the players might have gotten distracted by it and not paid attention to the narrator.

\insertPicture{0.45}{Screenshot_FirstRoom}{How the first four rooms of the game look.}

The last room is different from the others, as it is the last room. This room has the same dimensions and color as the other rooms, but there are no exits, only an entrance. In addition, there is a pile of coins in the middle of the room that represents the player's reward. Attempting to pick the coins up will end the game.

What the player is never told during the test, is that the first four rooms not only look alike, the are the exact same room. When the player choses an exit, the game records which side they chose and then sends them back to the entrance of the room. This repeats four times, after which the player is let into the final room with the reward. We decided to use this "one-room-fits-all" method, as there was no need to create fifteen different versions of the exact same room.

While we reuse the same room multiple times, we chose to only have the player go through the room a few times. There were two reasons for this:
\begin{enumerate}
	\item We did not want the player to become bored with walking through rooms, so we kept it to a low number. Adding more rooms would increase the likelihood that the player chose a door just for the sake of doing something different, thereby removing the point of having the narrator affect them.
	\item For every "layer" of rooms, the amount of rooms would be doubled. As we wanted unique script for every possible left/right-combination the player could take, we would need to create more than twice as much audio as we already had for every new layer added. This is a huge resource-cost and one we felt was not worth what little extra benefit we would gain.
\end{enumerate}

As we wanted a narrator to (attempt to) influence the player, we have him speak to the player every time they enter a room. In each room the narrator gives a hint to which door is the better choice, either saying that one would result in a better reward or saying that the other door would decrease the reward. This is part of trying to make the narrator influence the player's decision. 

The way the narrator talks to the player, implies that what the player is experiencing was something that happened in the past (\textit{"Taking the door on the right out of this room would grant him a much higher reward than taking the left door, but he had no way of knowing this."}).

What the narrator says specifically in each room, and how he reacts, depends on the route the player has chosen. This way, we avoid that some of the script may sound out of place because the player went an unexpected way.

The only room where the script does not change based on the route is the last room. We chose to do this, to give the player the feeling that the route they chose did not actually matter, as it actually did not matter. The reward is always the same, no matter what way the player chooses.

As the reward is always the same, no matter the route, it is clear that our narrator is unreliable, as he repeatedly tells the player that taking one path would be more beneficial to the player than taking the other. As the narrator in our game is aware that the reward will remain the same, he is not misinformed nor mistaken - he is lying and as so would be defined as an untrustworthy narrator using the amplified model suggested by Olson.

We decided to make the narrator unreliable because we wanted to see if the player could figure out that the narrator was lying, while only giving one, weak hit to that fact.

\subsection{Test Questions}
\label{Method_Questions}
To gather results, we used the "observe and interview" method. We observed the decisions that the players made in terms of which path they take through the game. Following the observation, we interview the players and ask them the questions presented below. In an effort not to influence the responses of the players, we have made the questions as objective and neutral as we possibly could. Additionally, we have attempted to not present the player with "yes/no" questions so that they are forced to explain their answer. In order to make it easier for the player to recognise the narrator as being untrustworthy, we have them play through the game once again (hopefully taking a different path) after they answer the first six questions.

\begin{itemize}

	\item \textit{"Why did you take the path you took?"} \\
 		This question would give us a clear idea of why the player made the decisions they made. Was it because of the reward the narrator talked about, was it just on impulse or was it to try to trick the narrator? \\

	\item \textit{"What did you think about the narrator?"} \\
		 This question gives us an indication of the how the player experienced the narrator; specifically how useful, reliable and trustworthy the narrator was. \\

	\item \textit{"In what way, if any, did the narrator influence the decisions you made in the game?"} \\
		 Because of our first research question, we are very interested in the influence the player felt from the narrator. This question might be partially answered by the first question, but we ask this question to make sure that we have a clear indication of the narrator's influence. \\

	\item \textit{"Would you do anything differently if you were to play again?"} \\
		 If the player is not interested in doing anything differently, it might be because the narrator had a lot of influence or none at all or anything in between. Depending on their path taken and their answer, it might be necessary to dig deeper into this answer. \\
	\item \textit{"What did you think about the different rooms you went through?"} \\
		 This question was for figuring if the player had even thought about the rooms or not, and if yes whether the rooms had any influence on the decisions of the player or not. \\

	\item \textit{"What influence did you feel the choices you made had on the outcome?"} \\
		 This question would give us an idea about whether the player felt that the narrator had been unreliable or not through the rooms. If the player felt their choices had no influence, we could deduce that the narrator had felt unreliable. \\

	\item \textit{"In what way, if any, did your opinion of the narrator change after the second play through?} \\
		  The motivation behind this question is to check if the player notices that the ending does not change even though they take a different path through the game.
\end{itemize}

We chose to not have more questions, as we felt these were enough to gather the information we were interested in. More questions might have yielded more information, but it would have been redundant as it would either cover what these questions cover, or be of no interest with regards to our research questions.

TODO: HOW TO TEST!?!!?!!?!!ONEONEONE

\section{Results}
\label{Results}
We tested eight people during this project. It is a small sample size and it means that the conclusions we make based on these results are not documented very well. We would have liked to test thirty or more so that we had a more diverse sample size.

As our primary research goal was to find out how much influence the narrator has on player choice, our focus was on what the participants in our tests said and did in terms of following what the narrator told them to.
\begin{table}
\begin{center}
\renewcommand{\arraystretch}{1.3}
\caption {How many followed the narrators instructions}
\label{table-followed_instructions}
\begin{tabular}{ | p{2.3cm} | p{2.3cm} | p{2.3cm} | }
  \hline                       
  \textbf{Followed} & \textbf{Bit of both} & \textbf{Went against} \\
  \hline  
  3 & 2 & 3 \\
  \hline  
  For the reward & Wanted reward, but became suspicious that the narrator was lying & Wanted to see the narrator’s reaction \\
  \hline  
  Because there was nothing else to show the way & Ignored the narrator in the beginning, but wanted the larger reward at the end. & Felt that something shady was going on \\
  \hline  
\end{tabular}
\end{center}
\end{table}

As shown on table \ref{table-followed_instructions}, it is clear that the players went many different ways when they played the game. Based on the answers, it seems that the more the players felt the narrator was unreliable, the more willing they were to go any which way they wanted.

What is important to note is that all the players were influenced by the narrator in one way or another. None of them decided to go a certain way because they did not care about what the narrator told them to. 

While the players went different ways in the tests, only one of them felt that the decisions made during the game influenced the ending and outcome of the game. From the responses of the playtesters, it is clear that it was because there was nothing to compare their actual reward with whichever reward they believed they would get. There was no point in the game where they were told what they would receive at the end, so when they encountered a pile of coins in the final room, it was not clear if the size of the pile had increased or decreased as a result of their actions and choices in the game.


Another thing we felt was important to look at, was how good the players were at identifying the narrator as unreliable. The table below shows how many players that figured it out after the first and the second playthrough.
\begin{table}
\begin{center}
\renewcommand{\arraystretch}{1.3}
\caption {How many identified the narrator as unreliable and when}
\label{table-unreliable_narrator}
\begin{tabular}{ | p{2.3cm} | p{2.3cm} | p{2.3cm} | }
  \hline                       
  \textbf{First playthrough} & \textbf{Second playthrough} & \textbf{Never} \\
  \hline  
  3 & 8 & 0 \\
  \hline  
\end{tabular} 
\end{center}
\end{table}

As shown by table \ref{table-unreliable_narrator}, less than half the players figured out that the narrator was unreliable after the first playthrough, and the ones that did were not entirely sure about it. They had a suspicion that something the narrator said was incorrect, but they were could not point out why. After the second playthrough, all of them had figured it out, however.

The players who were suspicious during the first playthrough did not base their suspicion on contradictions in the game. Instead, they felt the narrator was unreliable because there was nothing to indicate that he was reliable. For them it was a lack of confirmation of reliability, rather than contradictions, that made them not trust him.
\subsection{The influence of the narrator}

%What can we use the results for?
%RQ 1:
%Narrator has influence in some way (at least in this particular case)
%If player believes the narrator to be reliable and/or trustworthy, they will follow directions given by the narrator, unless they are tempted to experiment
%If there is any slight mistrust or any belief that the narrator is unreliable, the player is much more likely to experiment

\subsection{The unreliability of the unreliable narrator}

%RQ 2:
%results indicate that some players do not need a strong contradiction to get the feeling that the narrator is unreliable, but instead a lack of confirmation that their actions matter is enough
%offering choices that should supposedly change the state and/or the outcome of the game makes it much easier to realise/identify a narrator as unreliable because multiple playthroughs makes it easily evident


\section{Conclusions}
In this report, we have covered what constitutes a narrator and can make a narrator unreliable. We have also offered two hypothesises on how influential the narrator can be on the choices that players make in games. Our hypothesis are that the stronger the personal relation is between the narrator and the player and/or player character, the more influential the narrator is. Additionally, if the narrator is close to the narrative and the player and/or player character is strongly opinionated about the narrator, the narrator will have greater influence on player choice.

We use a simple game made in Unity to test our two research questions; 
\\(1) \textit{"What influence does the narrator have on decision making in games?"}, 
\\(2) \textit{"Does the player experience an untrustworthy narrator as reliable if almost no contradictions are given in the narrative?"}
\\The game consists of a series of rooms where the player has the choice of two doors. A distant third-person, untrustworthy narrator that is externally focalized on the player character to relate the narrative to the player.

Additionally, we have described the questions and the testing method we used to gather results we used to analyse and answer our research questions.

The results of our tests showed that players are highly influenced by what the narrator is relating to them and will in most cases either follow the instructions given by the narrator or try to do the opposite of what the narrator is asking in order to experiment with the game. We also conclude that while not all players will notice an unreliable narrator on the first playthrough, the possibility of playing through a game multiple times and making different choices in said playthroughs open up for the possibility of making identifying the narrator as unreliable only possible if the game is played more than once.

These conclusions leads us to suggest that game narrative and story writers keep in mind that if they want an influential narrator, they must make the player strongly opinionated about the narrator and/or make the relation with the narrator personal to the player and/or the player character.

\subsection{Improving the project}
There are many elements in the project that could be improved to increase the quality of the project if further work is done. We have listed them in the order of importance.

The number of people we tested was far too few to form a solid conclusion based on the feedback we received. Testing a minimum of thirty people would ensure that there should at least be a third of our testers that took the same path. Players might take the same path for different reasons, which is why this would be interesting. In addition, most of our testers were also used to playing games, so the feedback is biased towards more experienced gamers. This could possibly have the effect that the testers are more likely to try to "game" by testing the boundaries of what is possible\footnote{One of our testers actually did so - he spent a minute or two in one of the test rooms trying to find a different way out.}, instead of making one of the two choices given to them by the narrator. This is also more likely to happen because the test happens in a 3D-game environment instead of say a test-based environment.

The voice acting of the narrator could be improved significantly. We did it internally on the team and because we do not have great voice acting capabilities and below average quality recording equipment, we end up with poor quality in voice acting. An easy and cheap method of improving this would be to have subtitles available, although that could possibly detract from the tone of the narrator's voice.

Creating more varied game play mechanics would make the game have more variance and would likely make playing through a longer game less dull. This would lead to the opportunity of testing a more varied set of combinations in player choice and thus a better indication of the narrator's influence on the player.

The rooms in the game are completely bare and grey, which can lead to a monotonous player experience. Furnishing the rooms with simple furniture would not be very distracting from the experience and would make it so that the narrator is not the entire focus of the experience.

Using some of the research that Porteous et al. \cite{Por} did on character's point of view and narrative actions, it could be possible to observe how players react to the same given choice from different perspectives, e.g. a greedy or self-serving researcher attempting to direct the current protagonist to make the wrong choices to save money on his research budget.

\subsection{Who did what?}
Jakob Melnyk wrote the dialogue and did the voice acting. Jacob Grooss programmed the game and made the scenes in Unity. Equal work was done on writing the report.

\
% Trigger a \newpage just before a given reference number in order to
% balance the columns on the last page.  Adjust the value as needed;
% it may need to be readjusted if the document is modified later.
%\IEEEtriggeratref{8}
% The "triggered" command can be changed if desired:
%\IEEEtriggercmd{\enlargethispage{-5in}}

% The references section can either be generated by hand or by an
% automatic tool like BibTeX.  If using BibTex, use the standard IEEEtran
% bibliography style.
%\bibliographystyle{IEEEtran.bst}
%
% The argument to \bibliography is/are the name(s) of your BibTeX file(s)
% that contains string definitions and bibliography database(s).
%\bibliography{IEEEabrv,SamplePaper}
%
% If you generate the bibliography by hand, or if you copy in the
% resultant .bbl file, set the second argument of \begin to the number of
% references in the bibliography (used to reserve space for the reference
% number labels box).

\begin{thebibliography}{2}
\bibitem{Abbot}
H. Potter Abbot, \emph{The Cambridge Introduction to Narrative, Second Edition}.\hskip 1em plus 0.5em minus 0.4em\relax
  Cambridge University Press, 2008.
\bibitem{Bal}
M. Bal, \emph{Narratology: Introduction to the Theory of Narrative. Second Edition. Chapter 1}.\hskip 1em plus 0.5em minus 0.4em\relax
  University of Toronto Press Incorporated, 2004.
\bibitem{Olson}
G. Olson, ``Reconsidering Unreliability:Fallible and Untrustworthy Narrators,'' \emph{Narrative, Vol. 11, No. 1 (Jan., 2003)}, pp. 93--109, 2003.
\bibitem{Barthes}
R.Barthes, L. Duisuit, ``An Introduction to the Structural Analysis of Narrative,'' \emph{New Literary History
, Vol. 6, No. 2, On Narrative and Narratives. (Winter, 1975)}, pp. 237--272, 1975.
\bibitem{Porteous}
J. Porteous, M. Carvazza, F. Charles, ``Narrative Generation through Characters’ Point of View,'' \emph{Proc. of 9th Int. Conf.on Autonomous Agents and Multiagent Systems (AAMAS 2010), van der Hoek, Kaminka, Lespérance, Luck and Sen (eds.)}, pp. XXX-XXX, 2010.
\end{thebibliography}


% That's all folks...
\end{document}
