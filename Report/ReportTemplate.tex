\documentclass[conference]{IEEEtran}
% If the IEEEtran.cls has not been installed into the LaTeX system files,
% manually specify the path to it.  e.g.
% \documentclass[conference]{./IEEEtran}

% Add and required packages here
\usepackage{graphicx,times,amsmath}

% Correct bad hyphenation here
\hyphenation{op-tical net-works semi-conduc-tor IEEEtran}

% To create the author's affliation portion using \thanks
\IEEEoverridecommandlockouts

\textwidth 178mm
\textheight 239mm
\oddsidemargin -7mm
\evensidemargin -7mm
\topmargin -6mm
\columnsep 5mm

\begin{document}

% Project title: keep the \ \\ \LARGE\bf in it to leave enough margin.
\title{\ \\ \LARGE\bf Title of your Report}

\author{Jacob Claudius Grooss, Jakob Melnyk}

% Uncomment out the following line for invited papers
%\specialpapernotice{(Invited Paper)}

% Make the title area
\maketitle

\section{Introduction}
\label{Introduction}

This report has been written as part of a project in the Interactive Narrative course on the Master of Games at the IT-University of Copenhagen.\\

What effect does a narrator having on player choice in games? In this report, we will attempt to answer this question by reviewing the different definitions of a narrator presented by Mieke Bal and H. Porter Abbot and the different definitions by Wayne Booth, Ansgar N\"unning and Greta Olson of what makes a narrator unreliable in regards to the narrative that is being related. We will use the different definitions to define our use of the concept of an unreliable narrator and then address what effect they can have in a narrative (interactive or otherwise).

After determining what an unreliable narrator is, we will discuss the motivation behind the research questions presented in this report: What influence, if any, a narrator has on a player's decision-making in games and if a player can determine if a narrator is unreliable, given that there is only the player and the narrator. 

Next, we will discuss the method we used to gather the data we use to conclude on our research questions - a simple game made to test the influence that a narrator has on the player. We will discuss the thought process behind the concept of the game and the design choices we made in terms of length and variance in the game. In addition, we will explain the gameplay, the structure of the game, how the narrator acts in the game and why we chose the game length that we did. Then we will list the questions that we asked the players that tested our game and the motivation behind them.

Finally, we discuss the results of testing the game TODO: Get results and talk about them here..

\section{Related Work}
The definition of what makes a narrator unreliable is a highly contested point. In her essay on fallible and untrustworthy narrators, Greta Olson first analyses Wayne Booth's introduction of the term unreliable and Ansgar N\"unning's criticism of Booth's definition. In addition, she offers an amplification of Booth's model of unreliability\cite{Olson}. This work is strongly related to ours because we are interested in what effect that the unreliability of a narrator has on the decisions that the players make.

Porteous et al. \cite{Porteous} analysis of the influence that the point of view of the characters have on the narrative is interesting. The decision making of players could possibly be heavily influenced by which characters point of view the player is experiencing. In the game we made and tested for this paper, the narrator's influence on the player is heavily dependent on whether the player cares about what happens to the character they are playing. If they instead had the point of view of the narrator, they would know that the narrator is lying and such they would likely make different choices (or not care at all).

\section{Method}
\label{Method}

To answer our research question we decided to make a simple game, let people play it and ask them questions to obtain the information we were interested in.

%Here you describe what you did and argue for why you want to do it this way and not in some other way. All design choices need to be well motivated.%

\subsection{The Game}
\label{Method_Game}

The game we created required the player to go through four rooms, each time presenting the player with a choice between two doors; one on the right side of the room and one on the left side of the room. After the four rooms, the player ends in a final room with no exits and the game is over. The structure of the rooms can be seen in figure \ref{fig:RoomStructure}.

\begin{figure}[h!]
  \centering
\includegraphics[width=0.5\textwidth]{Parts/RoomStructure}
\caption{The structure of the rooms in the game.}
\label{fig:RoomStructure}
\end{figure}

The first four rooms the player goes through are identical, with an entrance and two exits. The room is square and the walls are grey. There are no objects in the rooms, as we only wanted the narrator to be influencing the player's decision. Had we placed anything else in the rooms, the players might have gotten distracted by it and not paid attention to the narrator.

The last room is different from the others, as it is the last room. This room has the same dimensions and color as the other rooms, but there are no exits, only an entrance. In addition, there is a pile of coins in the middle of the room that represents the player's reward. Attempting to pick the coins up will end the game.

As the player progresses through the rooms, the narrator will speak to them in a way that implies that what the player is experiencing was something that happened in the past (\textit{"Taking the door on the right out of this room would grant him a much higher reward than taking the left door, but he had no way of knowing this."}).

In each room the narrator gives a hint to which door is the better choice, either saying that one would result in a better reward or saying that the other door would decrease the reward. This is part of trying to make the narrator influence the player's decision.

What the narrator says and how he reacts depends on the route the player has chosen. This way, we avoid that some of the script may sound out of place because the player went an unexpected way.

In the last room, the script for the narrator is always the same, no matter the route. We chose to do this, to give the player the feeling that the route they chose did not actually matter, as it actually did not matter. The reward is always the same.

What the player is never told, is that the first four rooms are actually the exact same room. When the player choses an exit, the game records which side they chose and then sends them back to the entrance of the room. This repeats three times, after which the player is let into the final room with the reward. We chose this "one-room-fits-all" method, as there was no need to create fifteen different versions of the exact same room.

While we reuse the same room multiple times, we chose to only have the player go through a few rooms. There were two reasons for this:
\begin{enumerate}
	\item We did not want the player to become bored with walking through rooms, so we kept it to a low number. Adding more rooms would increase the likelihood that the player chose a door just for the sake of doing something different, thereby removing the point of having the narrator affect them.
	\item For every "layer" of rooms, the amount of rooms would be doubled. As we wanted unique script for every possible left/right-combination the player could go, we would need to create more than twice as much audio as we already had for every new layer added.
\end{enumerate}

\section{Results}
\label{Results}
We tested eight people during this project. It is a small sample size and it means that the conclusions we make based on these results are not documented very well. We would have liked to test thirty or more so that we had a more diverse sample size.

As our primary research goal was to find out how much influence the narrator has on player choice, our focus was on what the participants in our tests said and did in terms of following what the narrator told them to.
\begin{table}
\begin{center}
\renewcommand{\arraystretch}{1.3}
\caption {How many followed the narrators instructions}
\label{table-followed_instructions}
\begin{tabular}{ | p{2.3cm} | p{2.3cm} | p{2.3cm} | }
  \hline                       
  \textbf{Followed} & \textbf{Bit of both} & \textbf{Went against} \\
  \hline  
  3 & 2 & 3 \\
  \hline  
  For the reward & Wanted reward, but became suspicious that the narrator was lying & Wanted to see the narrator’s reaction \\
  \hline  
  Because there was nothing else to show the way & Ignored the narrator in the beginning, but wanted the larger reward at the end. & Felt that something shady was going on \\
  \hline  
\end{tabular}
\end{center}
\end{table}

As shown on table \ref{table-followed_instructions}, it is clear that the players went many different ways when they played the game. Based on the answers, it seems that the more the players felt the narrator was unreliable, the more willing they were to go any which way they wanted.

What is important to note is that all the players were influenced by the narrator in one way or another. None of them decided to go a certain way because they did not care about what the narrator told them to. 

While the players went different ways in the tests, only one of them felt that the decisions made during the game influenced the ending and outcome of the game. From the responses of the playtesters, it is clear that it was because there was nothing to compare their actual reward with whichever reward they believed they would get. There was no point in the game where they were told what they would receive at the end, so when they encountered a pile of coins in the final room, it was not clear if the size of the pile had increased or decreased as a result of their actions and choices in the game.


Another thing we felt was important to look at, was how good the players were at identifying the narrator as unreliable. The table below shows how many players that figured it out after the first and the second playthrough.
\begin{table}
\begin{center}
\renewcommand{\arraystretch}{1.3}
\caption {How many identified the narrator as unreliable and when}
\label{table-unreliable_narrator}
\begin{tabular}{ | p{2.3cm} | p{2.3cm} | p{2.3cm} | }
  \hline                       
  \textbf{First playthrough} & \textbf{Second playthrough} & \textbf{Never} \\
  \hline  
  3 & 8 & 0 \\
  \hline  
\end{tabular} 
\end{center}
\end{table}

As shown by table \ref{table-unreliable_narrator}, less than half the players figured out that the narrator was unreliable after the first playthrough, and the ones that did were not entirely sure about it. They had a suspicion that something the narrator said was incorrect, but they were could not point out why. After the second playthrough, all of them had figured it out, however.

The players who were suspicious during the first playthrough did not base their suspicion on contradictions in the game. Instead, they felt the narrator was unreliable because there was nothing to indicate that he was reliable. For them it was a lack of confirmation of reliability, rather than contradictions, that made them not trust him.
\subsection{The influence of the narrator}
As our results show, there is an influence on the player by the narrator. While it is natural to assume that some non-story-based games will have much less influence (perhaps none at all), most games with an amply sized narrative will see the narrator influence player decisions. 

The degree to which the narrator is influential appears to be based on how much focus the player puts on what the narrator is relating to the player. If the player does not care what the narrator is saying, then the narrator will have almost no influence compared to cases where a player is paying a lot of attention to what is being narrated.

Additionally, our results indicate that if the player believes the narrator to be reliable and/or trustworthy, then the player is more likely to follow instructions or directions given by the narrator. In contrast, if the player has any mistrust in the narrator, then the player is much more likely to experiment or go his own way.

These points leads us to suggest that game narrative/story writers take great care in how they present the narrator to the player as it can make a big difference in how the player makes decisions in the game. Additionally, we suggest that authors use a narrator that relates personally to the player (and/or player character) and/or use a narrator that is close to the narrative in order to have the narrator be influential on the the decision-making of the player.
\subsection{The unreliability of the unreliable narrator}
Our results indicate that players do not necessarily need a contradiction in order for them to doubt the reliability of the narrator. In our test game, the narrator is not contradicted at all (until the last room, where a tiny hint is given), but even so, a few players felt that because they were not shown that the narrator was reliable, he was instead unreliable. However, it was not until the second playthrough that people gave clear indication that they believed the narrator to be untrustworthy (and not just fallible).

Considering the test results, we believe the answer to our second research question is that the player can experience a narrator as, at least, unreliable without being shown a contradiction to any statement made by the narrator.

Based on what we saw in the results, we would suggest that further experimentation is done in relation to only showing the narrator being contradicted across multiple playthroughs. If the player believes that during the first playthrough the narrative is being related to him/her by a reliable source, it would be very interesting to observe the reactions of the player if during a second playthrough, he/she is given a clear contradiction in the narrative to make them doubt the reliability of the narrator.

\section{Conclusions}

\subsection{Improving the project}
There are many elements in the project that could be improved to increase the quality of the project if further work is done. We have listed them in the order of importance.

The number of people we tested was far too few to form a solid conclusion based on the feedback we received. Testing a minimum of thirty people would ensure that there should at least be a third of our testers that took the same path. Players might take the same path for different reasons, which is why this would be interesting. In addition, most of our testers were also used to playing games, so the feedback is biased towards more experienced gamers. This could possibly have the effect that the testers are more likely to try to "game" by testing the boundaries of what is possible\footnote{One of our testers actually did so - he spend several minutes in one of the test rooms to find a different way out.}, instead of making one of the two choices given to them by the narrator. This is also more likely to happen because the test happens in a 3D-game environment instead of say a test-based environment.

The voice acting of the narrator could be improved significantly. We did it internally on the team and because we do not have great voice acting capabilities and below average quality recording equipment, we end up with poor quality in voice acting. An easy and cheap method of improving this would be to have subtitles available, although that could possibly detract from the tone of the narrator's voice.

Creating more varied game play mechanics would make the game have more variance and would likely make playing through a longer game less dull. This would lead to the opportunity of testing a more varied set of combinations in player choice and thus a better indication of the narrator's influence on the player.

The rooms in the game are completely bare and grey, which can lead to a monotonous player experience. Furnishing the rooms with simple furniture would not be very distracting from the experience and would make it so that the narrator is not the entire focus of the experience.

Using some of the research that Porteous et al. \cite{Por} did on character's point of view and narrative actions, it could be possible to observe how players react to the same given choice from different perspectives, e.g. a greedy or self-serving researcher attempting to direct the current protagonist to make the wrong choices to save money on his research budget.

\subsection{Who did what?}
Jakob Melnyk wrote the dialogue and did the voice acting. Jacob Grooss programmed the game and made the scenes in Unity. Equal work was done on writing the report.

\
% Trigger a \newpage just before a given reference number in order to
% balance the columns on the last page.  Adjust the value as needed;
% it may need to be readjusted if the document is modified later.
%\IEEEtriggeratref{8}
% The "triggered" command can be changed if desired:
%\IEEEtriggercmd{\enlargethispage{-5in}}

% The references section can either be generated by hand or by an
% automatic tool like BibTeX.  If using BibTex, use the standard IEEEtran
% bibliography style.
%\bibliographystyle{IEEEtran.bst}
%
% The argument to \bibliography is/are the name(s) of your BibTeX file(s)
% that contains string definitions and bibliography database(s).
%\bibliography{IEEEabrv,SamplePaper}
%
% If you generate the bibliography by hand, or if you copy in the
% resultant .bbl file, set the second argument of \begin to the number of
% references in the bibliography (used to reserve space for the reference
% number labels box).

\begin{thebibliography}{2}
\bibitem{Abbot}
H. Potter Abbot, \emph{The Cambridge Introduction to Narrative, Second Edition}.\hskip 1em plus 0.5em minus 0.4em\relax
  Cambridge University Press, 2008.
\bibitem{Bal}
M. Bal, \emph{Narratology: Introduction to the Theory of Narrative. Second Edition. Chapter 1}.\hskip 1em plus 0.5em minus 0.4em\relax
  University of Toronto Press Incorporated, 2004.
\bibitem{Olson}
G. Olson, ``Reconsidering Unreliability:Fallible and Untrustworthy Narrators,'' \emph{Narrative, Vol. 11, No. 1 (Jan., 2003)}, pp. 93--109, 2003.
\bibitem{Barthes}
R.Barthes, L. Duisuit, ``An Introduction to the Structural Analysis of Narrative,'' \emph{New Literary History
, Vol. 6, No. 2, On Narrative and Narratives. (Winter, 1975)}, pp. 237--272, 1975.
\bibitem{Porteous}
J. Porteous, M. Carvazza, F. Charles, ``Narrative Generation through Characters’ Point of View,'' \emph{Proc. of 9th Int. Conf.on Autonomous Agents and Multiagent Systems (AAMAS 2010), van der Hoek, Kaminka, Lespérance, Luck and Sen (eds.)}, pp. XXX-XXX, 2010.
\end{thebibliography}


% That's all folks...
\end{document}
